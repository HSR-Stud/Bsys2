\section{Virtueller Speicher}

Jeder Prozess hat scheinbar seinen Adressraum für sich alleine zur Verfügung
(teilt ihn nur mit dem Betriebssystem).

\subsection{Segmentbasierte Adressumsetzung}

Jede Region der Prozessadressraumverwaltung wird in Segmente aufgeteilt. Pro
Segment wird eine gemeinsame Umsetzungsregel definiert.

\textbf{Abbildungsfunktion:} $A_P = A_L + A_{ik}$

\begin{description}
	\item[$A_P$] Speicheradresse (physische Adresse)
	\item[$A_L$] Programmadresse (logische bzw. virtuelle Adresse)
	\item[$A_{ik}$] Segmentstartadresse
	\item[$i$] Prozessnummer
	\item[$k$] Segmentnummer
\end{description}

\subsection{Seitenbasierte Adressumsetzung}

Unterteile Hauptspeicher in Reihe gleich grosser Standardblöcke (=Seiten).
Reserviere für Regionen eines Prozesses eine ausreichende Anzahl Seiten.

\textbf{Abbildungsfunktion:} $A_P = A_L + v_{ij} \cdot S$

\begin{description}
	\item[$A_P$] Speicheradresse (physische Adresse)
	\item[$A_L$] Programmadresse (logische bzw. virtuelle Adresse)
	\item[$S$] Seitengrösse (von CPU-Typ abhängig)
	\item[$v$] Verschiebungsfaktor
	\item[$i$] Prozessnummer
	\item[$j$] Seitennummer
\end{description}

\subsection{Segment- und Seitenbasierte Adressumsetzung}

TODO
